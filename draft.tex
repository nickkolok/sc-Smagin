\documentclass[a4paper,14pt]{article} %размер бумаги устанавливаем А4, шрифт 12пунктов
\usepackage[T2A]{fontenc}
\usepackage[utf8]{inputenc}
\usepackage[english,russian]{babel} %используем русский и английский языки с переносами
\usepackage{amssymb,amsfonts,amsmath,mathtext,cite,enumerate,float,amsthm} %подключаем нужные пакеты расширений
\usepackage[unicode,colorlinks=true,citecolor=black,linkcolor=black]{hyperref}
%\usepackage[pdftex,unicode,colorlinks=true,linkcolor=blue]{hyperref}
\usepackage{indentfirst} % включить отступ у первого абзаца
\usepackage[dvips]{graphicx} %хотим вставлять рисунки?
\graphicspath{{illustr/}}%путь к рисункам

\makeatletter
\renewcommand{\@biblabel}[1]{#1.} % Заменяем библиографию с квадратных скобок на точку:
\makeatother %Смысл этих трёх строчек мне непонятен, но поверим "Запискам дебианщика"

\usepackage{geometry} % Меняем поля страницы.
\geometry{left=2cm}% левое поле
\geometry{right=1cm}% правое поле
\geometry{top=2cm}% верхнее поле
\geometry{bottom=2cm}% нижнее поле

\renewcommand{\theenumi}{\arabic{enumi}}% Меняем везде перечисления на цифра.цифра
\renewcommand{\labelenumi}{\arabic{enumi}}% Меняем везде перечисления на цифра.цифра
\renewcommand{\theenumii}{.\arabic{enumii}}% Меняем везде перечисления на цифра.цифра
\renewcommand{\labelenumii}{\arabic{enumi}.\arabic{enumii}.}% Меняем везде перечисления на цифра.цифра
\renewcommand{\theenumiii}{.\arabic{enumiii}}% Меняем везде перечисления на цифра.цифра
\renewcommand{\labelenumiii}{\arabic{enumi}.\arabic{enumii}.\arabic{enumiii}.}% Меняем везде перечисления на цифра.цифра



% https://tex.stackexchange.com/questions/69085/two-different-calligraphic-font-styles-in-math-mode
\usepackage{calrsfs}
\DeclareMathAlphabet{\pazocal}{OMS}{zplm}{m}{n}

\sloppy


\renewcommand\normalsize{\fontsize{14}{25.2pt}\selectfont}

\begin{document}
% !!!
% Здесь начинается реальный ТеХ-код
% Всё, что выше - беллетристика

\subsection{Определение и свойства оператора ортогонального проектирования}

Пусть задано гильбертово пространство $H$ и подпространство $L \subset H$.
Тогда $H$ представимо в виде ортогональной суммы $H = L \oplus L^\perp$, то есть
всякий $u \in H$ по теореме о проекциях однозначно представим в виде суммы
$u = u_1 + u_2$, где $u_1 \in L$ и $u_2 \in L^\perp$.
Определим оператор $P : H \to L \subset H$
правилом $Pu = u_1$.
Построенный оператор $P$ называется ортогональным проектором пространства $H$ на подпространство $L$,
или просто ортопроектором $H$ на $L$.

Свойства ортопроектора $P$:
\begin{itemize}
	\item
		$P$ линейный, ограниченный и самосопряжённый
	\item
		$P^2 = P$
	\item
		$I-P$ есть ортопроектор $H$ на $L^\perp$
\end{itemize}

\subsection{Лемма 3.1 об элементе наилучшей аппроксимации в подпространстве}

Пусть $H$~--- гильбертово пространство,
$L$~--- подпространство $H$,
$P$~--- оператор ортогонального проектирования $H$ на $L$.
Тогда для всякого элемента $u \in H$
\begin{equation}
	\min_{v\in L} \|u - v\|_H = \|u - P u\|_H,
\end{equation}
то есть элемент $Pu \in L$ наилучшим образом аппроксимирует $u \in H$.

\subsection{Задача определения проекции $P_n\in L_n = \mathcal{L}(e_1,...,e_n)$ (проектирования на конечномерное подпространство)}

Пусть в гильбертовом пространстве $H$ задана линейно независимая система элементов
$\{e_1, e_2 , ... , e_n \}$.
Рассмотрим конечномерную линейную оболочку
$L_n = \mathcal{L} (e_1 , e_2 , ... , e_n )$,
которая будет конечномерным подпространством $H$.
Пусть $P_n$~--- ортопроектор $H$ на $L_n$.

Задача: для элемента $u \in H$ определить $P_n u \in L_n$.

Заметим, что $P_n u = \sum_{k=1}^{n} \alpha_k e_k$,
то есть следует определить коэффициенты
$\alpha_1, \alpha_2 , ... , \alpha_n$.

Из условия $P_n u - u \perp e_j$ выводится система:
\begin{equation}
	\sum_{k=1}^{n} \alpha_k (e_k, e_j)_H = (u, e_j)_H, ~~ j=1,...,n
	,
\end{equation}
которая всегда разрешима.

\subsection{Численный алгоритм построения приближенного решения по методу Ритца на конечномерном подпространстве гильбертова пространства (система уравнений)}

Пусть $H$~--- вещественное гильбертово пространство и в нем задан
$A : D(A) \subset H \to H$~--- положительно определенный оператор,
$H_A$~--- энергетическое пространство, порожденное этим оператором.
Пусть в $H_A$ задана линейно независимая система элементов $\{\omega_k\}_{k=1}^n$.
Обозначим линейную оболочку этих элементов $H_n = \mathcal{L}(\omega_1, \omega_2 , ... , \omega_n ) \subset H_A$.
Заметим, что $H_n$ является конечномерным подпространством пространства $H_A$.
Пусть $Q_n$ есть ортопроектор $H_A$ на $H_n$.

Пусть $u^*$ есть решение уравнения $Au=f$.
Тогда приближенное решение $Q_n u^* = u^*_n = \sum_{k=1}^n c_k \omega_k$ находится из системы
\begin{equation}
	\sum_{k=1}^{n} c_k (\omega_k, \omega_j)_A = (f, \omega_j)_H, ~~ j=1,...,n
	,
\end{equation}
которая всегда разрешима.


\subsection{Определение предельно плотной системы подпространств}

Кратко: последовательность подпространств гильбертова пространства называется предельно плотной,
если ортопроекторы на эти подпространства сильно сходятся к тождественному оператору.

Подробно:

Пусть $H$~--- вещественное гильбертово пространство и в нем задан
$A : D(A) \subset H \to H$~--- положительно определенный оператор,
$H_A$~--- энергетическое пространство, порожденное этим оператором.
Пусть в $H_A$ задана линейно независимая система элементов $\{\omega_k\}_{k=1}^n$.
Обозначим линейную оболочку этих элементов $H_n = \mathcal{L}(\omega_1, \omega_2 , ... , \omega_n ) \subset H_A$.
Заметим, что $H_n$ является конечномерным подпространством пространства $H_A$.

Пусть теперь задана последовательность подпространств
$\{H_n \} \subset H_A$ такая, что $\|(Q_n - I)v\|_A \to 0$ при
$n \to \infty$ для любого $v \in H_A$.
Такую последовательность подпространств $\{H_n \}$
будем называть предельно плотной в пространстве $H_A$.
В таком случае
\begin{equation}
	\|u^*_n - u^*\|_A = \|(Q n - I)u^* \|_A \to 0,
	~~~
	n\to\infty
\end{equation}
где $u^*$~--- обобщенное решение уравнения $Au = f$,
а $u^*_n$~--- приближенное решение этого уравнения.


\subsection{Теорема 3.1 о минимуме расширенного функционала энергии на подпространстве гильбертова пространства}

Пусть $H$~--- вещественное гильбертово пространство и в нем задан
$A : D(A) \subset H \to H$~--- положительно определенный оператор,
$H_A$~--- энергетическое пространство, порожденное этим оператором.
Пусть в $H_A$ задана линейно независимая система элементов $\{\omega_k\}_{k=1}^n$.
Обозначим линейную оболочку этих элементов $H_n = \mathcal{L}(\omega_1, \omega_2 , ... , \omega_n ) \subset H_A$.
Заметим, что $H_n$ является конечномерным подпространством пространства $H_A$.

Рассмотрим на элементах $u_n \in H_n$ расширенный функционал энергии
\begin{equation}
	\tilde{F}(u_n ) = \|u_n \|^2_A - 2(u_n, f )_H ,
	~~~
	f \in H.
\end{equation}

Теорема 3.1.
Функционал $\tilde{F}(u_n)$ на подпространстве $H_n$ достигает
минимальное значение на элементе $u^*_n = Q_n u^*$,
где $u^*$~--- обобщенное решение уравнения $Au = f$, а $Q_n$ есть ортопроектор $H_A$ на $H_n$.


\subsection{Определение полной системы элементов и теорема 3.2 о предельно плотной последовательности подпространств}

Пусть $H$~--- гильбертово пространство.
Система элементов $\{e_i \}_{i=1}^\infty \subset H$
называется полной в пространстве $H$, если любой конечный набор $\{e_i \}_{i=1}^k$ линейно независим
и
\begin{equation}
	\forall(\varepsilon > 0)
	\forall(u\in H)
	\exists(N\in\mathbb{N})
	\exists(c_1, c_2, ..., c_N)
	\left[
		\left\|u-\sum_{k=1}^N c_k e_k\right\| < \varepsilon
	\right]
	.
\end{equation}


Теорема 3.2.
Пусть $H$~--- гильбертово пространство и
система элементов $\{e_i \}_{i=1}^\infty \subset H$ полна.
Положим теперь $H_n$ = $\mathcal{L} (e_1 , e_2 , ... , e_n )$ для $n \in \mathbb{N}$.
Тогда система подпространств $\{H_n \}$
является предельно плотной в пространстве $H$.


\subsection{Алгебраическая координатная система для нахождения приближенного решения первой краевой задачи}

Пусть поставлена первая краевая задача на отрезке $[a,b]$.

TODO: можно бы и написать КЗ... Но лень.

Положим $\omega_i(x) = (x-a)(x-b)z_{i-1}$,
где $z_i$~--- многочлен степени точно $i$.
Положим $H_n = \mathcal{L}(\omega_1, ..., \omega_n)$.
Так как система $\{\omega_i\}$ полна в энергетическом пространстве $\overset{\circ}{W^1_2}$,
то последовательность $\{H_n\}$ предельно плотна в $\overset{\circ}{W^1_2}$ и, следовательно,
может быть использована для построения приближенных решений.

\subsection{Тригонометрическая координатная система для нахождения приближенного решения первой краевой задачи}

Пусть поставлена первая краевая задача на отрезке $[a,b]$.

TODO: можно бы и написать КЗ... Но лень.

Положим $\omega_j(x) = \sin\frac{j\pi(x-a)}{b-a}$,
где $j\in\mathbb{N}$.

Положим $H_n = \mathcal{L}(\omega_1, ..., \omega_n)$.
Так как система $\{\omega_j\}$ полна в энергетическом пространстве $\overset{\circ}{W^1_2}$,
то последовательность $\{H_n\}$ предельно плотна в $\overset{\circ}{W^1_2}$ и, следовательно,
может быть использована для построения приближенных решений.


\subsection{Тригонометрическая координатная система для нахождения приближенного решения второй и третьей краевой задач}

Пусть поставлена вторая или третья краевая задача на отрезке $[a,b]$.

TODO: можно бы и написать КЗ... Но лень.

Положим $\omega_j(x) = \cos\frac{j\pi(x-a)}{b-a}$,
где $j\in\mathbb{N}\cup\{0\}$.

Положим $H_n = \mathcal{L}(\omega_1, ..., \omega_n)$.
Так как система $\{\omega_j\}$ полна в энергетическом пространстве $W^1_2$,
то последовательность $\{H_n\}$ предельно плотна в $W^1_2$ и, следовательно,
может быть использована для построения приближенных решений.



\subsection{Кусочно-линейные координатные функции и лемма 3.2 об аппроксимации (две оценки)}

Пусть поставлена краевая задача на отрезке $[a,b]$.

Зафиксируем $n \in \mathbb{N}$.
Положим $h=\frac{b-a}{n}$.
Положим
\begin{equation}
	\omega(x) = \begin{cases}
		0,   & x \leq -1,
		\\
		1+x, & -1 \leq x \leq 0,
		\\
		1-x, &  0 \leq x \leq 1,
		\\
		0,   & 1 \leq x
		.
	\end{cases}
\end{equation}
Положим $\omega^{(n)}_j(x) = \omega\left(\frac{x-a-jh}{h}\right)$.

Положим $H_n^{(2,3)} = \mathcal{L}\left(\omega^{(n)}_0, ..., \omega^{(n)}_n\right)$
для второй и третьей краевых задач и
$H_n^{(1)} = \mathcal{L}\left(\omega^{(n)}_1, ..., \omega^{(n)}_{n-1}\right)$
для первой.

Тогда системы подпространств $\{H_n^{(1)}\}$ и $\{H_n^{(2,3)}\}$ предельно плотны в соотв. энергетических пространствах
($W^1_2$ для второй и третьей краевых задач; $\overset{\circ}{W^1_2}$ для первой).

Лемма 3.2.
Для любой функции $u(x)\in W^2_2$ и для любого $n\in\mathbb{N}$ найдётся такая функция $u_n(x)\in H^{(2,3)}_n$,
что
\begin{equation}
	\| u  - u _n\|_{L_2} \leq h^2 \|u''\|_{L_2}
	,
	~~~~~
	\| u' - u'_n\|_{L_2} \leq h   \|u''\|_{L_2}
	.
\end{equation}
Если же дополнительно $u(a)=u(b)=0$, то $u_n\in H^{(1)}_n$.

\subsection{Теорема 3.7 об оценке второй производной обобщённого решения краевой задачи}

Теорема 3.7. Пусть $u^*(x)$~--- обобщённое решение первой,
второй или третьей краевой задачи $Au = f$
с положительно определенным оператором $A$.
Тогда справедлива оценка
\begin{equation}
	\left\|
		\frac{d^2 u^*}{dx^2}
	\right\|_{L_2}
	\leq K \|f\|_{L_2}
	.
\end{equation}


\subsection{Теорема 3.8 об оценке погрешности приближенного решения краавой задачи в норме $L_2[a,b]$}

\subsection{Постановка вариационной задачи: форма, условия, искомый объект и теорема 4.1 Лакса--Мильграма}

\subsection{Формулировка задачи нахождения приближенного решения вариационной задачи по методу Галёркина}

\subsection{Теорема 4.3 (Сеа) об оценке погрешности приближенного решения вариационной задачи}


\end{document}
