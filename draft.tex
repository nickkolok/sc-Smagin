\documentclass[a4paper,14pt]{article} %размер бумаги устанавливаем А4, шрифт 12пунктов
\usepackage[T2A]{fontenc}
\usepackage[utf8]{inputenc}
\usepackage[english,russian]{babel} %используем русский и английский языки с переносами
\usepackage{amssymb,amsfonts,amsmath,mathtext,cite,enumerate,float,amsthm} %подключаем нужные пакеты расширений
\usepackage[unicode,colorlinks=true,citecolor=black,linkcolor=black]{hyperref}
%\usepackage[pdftex,unicode,colorlinks=true,linkcolor=blue]{hyperref}
\usepackage{indentfirst} % включить отступ у первого абзаца
\usepackage[dvips]{graphicx} %хотим вставлять рисунки?
\graphicspath{{illustr/}}%путь к рисункам

\makeatletter
\renewcommand{\@biblabel}[1]{#1.} % Заменяем библиографию с квадратных скобок на точку:
\makeatother %Смысл этих трёх строчек мне непонятен, но поверим "Запискам дебианщика"

\usepackage{geometry} % Меняем поля страницы.
\geometry{left=2cm}% левое поле
\geometry{right=1cm}% правое поле
\geometry{top=2cm}% верхнее поле
\geometry{bottom=2cm}% нижнее поле

\renewcommand{\theenumi}{\arabic{enumi}}% Меняем везде перечисления на цифра.цифра
\renewcommand{\labelenumi}{\arabic{enumi}}% Меняем везде перечисления на цифра.цифра
\renewcommand{\theenumii}{.\arabic{enumii}}% Меняем везде перечисления на цифра.цифра
\renewcommand{\labelenumii}{\arabic{enumi}.\arabic{enumii}.}% Меняем везде перечисления на цифра.цифра
\renewcommand{\theenumiii}{.\arabic{enumiii}}% Меняем везде перечисления на цифра.цифра
\renewcommand{\labelenumiii}{\arabic{enumi}.\arabic{enumii}.\arabic{enumiii}.}% Меняем везде перечисления на цифра.цифра



% https://tex.stackexchange.com/questions/69085/two-different-calligraphic-font-styles-in-math-mode
\usepackage{calrsfs}
\DeclareMathAlphabet{\pazocal}{OMS}{zplm}{m}{n}

\sloppy


\renewcommand\normalsize{\fontsize{14}{25.2pt}\selectfont}

\begin{document}
% !!!
% Здесь начинается реальный ТеХ-код
% Всё, что выше - беллетристика

\subsection{Определение и свойства оператора ортогонального проектирования}

\subsection{Лемма 3.1 об элементе наилучшей аппроксимации в подпространство}

\subsection{Задача определения проекции $P_n\in L_n = \mathcal{L}(e_1,...,e_n)$ (проектирования на конечномерное подпространство)}

\subsection{Численный алгоритм построения приближенного решения по методу Ритца на конечномерном подпространстве гильбертова пространства (система уравнений)}

\subsection{Определение предельно плотной системы подпространств}

\subsection{Теорема 3.1 о минимуме расширенного функционала энергии на подпространстве гильбертова пространства}

\subsection{Определение полной системы элементов и теорема 3.2 о предельно плотной последовательности подпространств}

\subsection{Алгебраическая координатная система для нахождения приближенного решения первой краевой задачи}

\subsection{Тригонометрическая координатная система для нахождения приближенного решения первой краевой задачи}
Синусы.

\subsection{Тригонометрическая координатная система для нахождения приближенного решения второй и третьей краевой задач}
Косинусы.

\subsection{Кусочно-линейные координатные функции и лемма 3.2 об аппроксимации (две оценки)}

\subsection{Теорема 3.7 об оценке второй производной обобщённого решения краевой задачи}

\subsection{Теорема 3.8 об оценке погрешности приближенного решения краавой задачи в норме $L_2[a,b]$}

\subsection{Постановка вариационной задачи: форма, условия, искомый объект и теорема 4.1 Лакса--Мильграма}

\subsection{Формулировка задачи нахождения приближенного решения вариационной задачи по методу Галёркина}

\subsection{Теорема 4.3 (Сеа) об оценке погрешности приближенного решения вариационной задачи}


\end{document}
